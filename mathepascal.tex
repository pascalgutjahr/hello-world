\documentclass{scrartcl}

\usepackage[aux]{rerunfilecheck}

\usepackage{polyglossia}
\setmainlanguage{german}

\usepackage{amsmath}
\usepackage{amssymb}
\usepackage{mathtools}

\usepackage{fontspec}

\usepackage[
  math-style=ISO,
  bold-style=ISO,
  sans-style=italic,
  nabla=upright,
  partial=upright,
]{unicode-math}

\usepackage[unicode]{hyperref}
\usepackage{bookmark}
%Einstellungen hier, z.b. Fonts

\newcommand{\be}{\begin{equation}} %Kurzbefehl für \begin{equation}
\newcommand{\ee}{\end{equation}} %Kurzbefehl für \end{equation}
%Befehle eingefügt um Zeichen zu sparen



\begin{document}

\section{Biot-Savart}
  Das Magnetfeld $\vec{B}$ am Ort $\vec{r}$ eines
  stromdurchflossenen Leiters
  ergibt sich zu
\begin{equation}
  \vec{B}(\vec{r})= \frac{\mu_0}% Zähler
  {4\symup{\pi}}% Nenner
  \int_V \vec{\jmath}(\vec{r}') \times \frac{\vec{r}-\vec{r}'}
  {\lvert \vec{r} - \vec{r}'\rvert^3}
  \symup{d}V'.
\end{equation}
Hierbei bezeichnet $\vec{\jmath}$ die Stromdichte am Ort $\vec{r}'$
 und $\mu_0$ die magnetische Feldkonstante.

 \section{Fehlerfortpflanzung}
 \begin{equation}
\sigma_k = \sqrt{\sum_{i=1}^N \Biggl(\frac{\partial{f}}{\partial{x_i}}
\sigma_i\Biggr)^2}
 \end{equation}

 \section{Maxwell-Gleichungen}
 \begin{align}
\nabla\cdot\vec{E} &= \frac{\rho}{\epsilon_0} &
\nabla\cdot\vec{B} &= 0 \\ % Zeilenumbruch
\nabla\times\vec{E} &= -\partial_t{\vec{B}} &
\nabla\times\vec{B} &= \mu_0\vec{\jmath} + \mu_0\epsilon_0\partial_t{\vec{E}}
 \end{align}

 \section{Wellengleichung}
Im Vakuum gelten $\rho=0$ und $\vec{\jmath}=0$, womit sich die
Maxwellgleichungen zu
 \begin{align}
\nabla\cdot\vec{E} &= 0 \\
\nabla\cdot\vec{B} &= 0 \\
\nabla\times\vec{E} &= -\partial_t\vec{B}\\
\nabla\times\vec{B} &= \mu_0\epsilon_0\partial_t{\vec{E}}
 \end{align}



\end{document}
