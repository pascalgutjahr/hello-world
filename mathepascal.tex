\documentclass{scrartcl}

\usepackage[aux]{rerunfilecheck}

\usepackage{polyglossia}
\setmainlanguage{german}

\usepackage{amsmath}
\usepackage{amssymb}
\usepackage{mathtools}

\usepackage{fontspec}

\usepackage[
  math-style=ISO,
  bold-style=ISO,
  sans-style=italic,
  nabla=upright,
  partial=upright,
]{unicode-math}

\usepackage[unicode]{hyperref}
\usepackage{bookmark}
%Einstellungen hier, z.B. Fonts

\newcommand{\be}{\begin{equation}} %Kurzbefehl für \begin{equation}
\newcommand{\ee}{\end{equation}} %Kurzbefehl für \end{equation}
%Befehle eingefügt um Zeichen zu sparen



\begin{document}

\section{Biot-Savart}
  Das Magnetfeld $\vec{B}$ am Ort $\vec{r}$ eines
  stromdurchflossenen Leiters
  ergibt sich zu
\begin{equation}
  \vec{B}(\vec{r})= \frac{\mu_0}% Zähler
  {4\symup{\pi}}% Nenner
  \int_V \vec{\jmath}(\vec{r}') \times \frac{\vec{r}-\vec{r}'}
  {\lvert \vec{r} - \vec{r}'\rvert^3}
  \symup{d}V'.
\end{equation}
Hierbei bezeichnet $\vec{\jmath}$ die Stromdichte am Ort $\vec{r}'$
 und $\mu_0$ die magnetische Feldkonstante.

 \section{Fehlerfortpflanzung}
 \begin{equation}
\sigma_k = \sqrt{\sum_{i=1}^N \Biggl(\frac{\partial{f}}{\partial{x_i}}
\sigma_i\Biggr)^2}
 \end{equation}

 \section{Maxwell-Gleichungen}
 \begin{align}
\nabla\cdot\vec{E} &= \frac{\rho}{\epsilon_0} &
\nabla\cdot\vec{B} &= 0 \\ % Zeilenumbruch
\nabla\times\vec{E} &= -\partial_t{\vec{B}} &
\nabla\times\vec{B} &= \mu_0\vec{\jmath} + \mu_0\epsilon_0\partial_t{\vec{E}}
 \end{align}

 \section{Wellengleichung}
Im Vakuum gelten $\rho=0$ und $\vec{\jmath}=0$, womit sich die
Maxwellgleichungen zu
 \begin{align}
\nabla\cdot\vec{E} &= 0     \label{eqn:max1}\\
\nabla\cdot\vec{B} &= 0   \label{eqn:max2}  \\
\nabla\times\vec{E} &= -\partial_t\vec{B} \label{eqn:max3} \\
\nabla\times\vec{B} &= \mu_0\epsilon_0\partial_t{\vec{E}} \label{eqn:max4}
 \end{align}
 reduzieren. Nach erneuter Rotation auf \eqref{eqn:max3} ergibt sich
 \begin{align}
   \nabla\times(\nabla\times\vec{E}) = \nabla\times(-\partial_t\vec{B})
   \label{eqn:9}
 \end{align}
 Nach dem Satz von Schwarz lassen sich die partiellen Ableitungen
 vertauschen, was zu
 \begin{align}
   \nabla\times(\nabla\times\vec{E}) = -\partial_t(\nabla\times\vec{B})
   \label{eqn:10}
 \end{align}
 führt. Wir setzen auf der rechten Seite \eqref{eqn:max4} ein:
 \begin{align}
   \nabla\times(\nabla\vec{E})=-\mu_0\epsilon_0\partial_t^2\vec{E}
   \label{eqn:11}
 \end{align}
 aus der linken Seite wird mit
 \begin{align}
 \nabla\times(\nabla\vec{E})= \nabla\cdot(\nabla\vec{E})- \Delta\vec{E}
 \label{eqn:12}
 \end{align}
 und ausnutzen von \eqref{max1}
 \begin{align}
   -\Delta\vec{E}=-\mu_0\epsilon_0\partial_t^2\vec{E}
   \label{eqn:13}
 \end{align}
 Dies ist die Wellengleichung für das elekrtische Feld,
 in der sich die Lichtgeschwindigkeit
 \begin{align}
 c=\frac{1}{\sqrt{\mu_0\epsilon_0}} \label{eqn:14}
\end{align}
identifizieren lässt. Dazu können wir
\begin{align}
  \biggl(\Delta-\frac{1}{c^2}\frac{\partial^2}{\partial_t^2}\biggr)\vec{E}=0
\end{align}
\label{eqn:15}
schreiben.

\section{Wellengleichung}
Ebene Welle:
\begin{align}
  \nabla^2 A - \frac{1}{c^2}\frac{\partial^2}{\partial t^2}A= 0
\end{align}
eine Lösung:

\begin{equation}
  A=A_0\exp (i({\symbf{kx}-\omega t}))  %symbf für fett schreiben
\end{equation}

Gruppen- und Phasengeschwindigkeit:

\begin{align}
  v_{Gr} &=\frac{\partial\omega}{\partial k} &
  v_{Ph} &= \frac{\omega}{k}
\end{align}

\section{Multipolentwicklung}
\begin{equation}
  \Phi(r) = \frac{1}{4\symup{pi}\epsilon_0}\Biggl(\frac{Q}{r} +
  \frac{\symbf{r\cdot p}} {r^3} + \frac{1}{2} \sum_{k,l}
  Q_{kl}\frac{r_k r_l}{r^5} + \cdot\cdot\cdot \Biggr),
\end{equation} \\
wobei
\begin{equation*}
  Q_{kl}= \sum_{i=1}^n q_i (3r_{ik} r_{il} - r_i^2 \delta_{kl})
\end{equation*}
\section{Jacobi-Matrix}

\begin{equation}
  \symbf{J}=
  \begin{pmatrix}
    \frac{\partial f_1}{\partial x_1} & \cdots & \frac{f_1}{\partial x_n} \\
    \vdots & \ddots & \vdots \\
    \frac{\partial f_m} {\partial x_1} & \cdots & \frac{\partial f_m}{\partial x_n}
  \end{pmatrix}
\end{equation}

\section{Harmonischer Oszillator}
\begin{equation}
  \ddot{x} + 2\gamma\dot{x} +\omega_0^2x= 0
\end{equation}
Reelle Lösung:
\begin{equation}
  x(t)= e^{-\gamma t}
(A\cos{(\omega t)} + B\sin{(\omega t)})
\end{equation}
mit
\begin{equation}
  \omega = \sqrt{\omega_0^2-\gamma^2}.
\end{equation}

4. Maxwell: \eqref{eqn:max4}

\end{document}
