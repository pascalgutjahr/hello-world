\documentclass[titlepage=firstiscover]{scrartcl}
%Mathematischer Header
\usepackage{scrhack}
\usepackage[aux]{rerunfilecheck}

\usepackage{polyglossia}
\setmainlanguage{german}

\usepackage{amsmath}
\usepackage{amssymb}
\usepackage{mathtools}

\usepackage{fontspec}

\usepackage[
  math-style=ISO,
  bold-style=ISO,
  sans-style=italic,
  nabla=upright,
  partial=upright,
]{unicode-math}
\usepackage[
locale=DE,
separate-uncertainty=true, % Immer Fehler mit ±
per-mode=symbol-or-fraction, % m/s im Text, sonst \frac
% alternativ:
% per-mode=reciprocal, % m s^{-1}
% output-decimal-marker=., % . statt , für Dezimalzahlen
]{siunitx}

\usepackage[section, below]{placeins}
\usepackage{caption}
\usepackage{graphicx}
\usepackage{grffile}
\usepackage{subcaption}
\usepackage{booktabs}
\usepackage{float}
\usepackage{figure}{htbp}
\usepackage{table}{htpb}
%Einstellungen hier, z.b. Fonts

\usepackage[unicode]{hyperref}
\usepackage{bookmark}

\newcommand{\be}{\begin{equation}} %Kurzbefehl für \begin{equation}
\newcommand{\ee}{\end{equation}}


\begin{document}
\title{Arbeiten mit \LaTeX}
\author{Julian Jung and Pascal Gutjahr}
\maketitle
\tableofcontents
\newpage
\section{Erste Schritte}
\subsection{Was ist \LaTeX}
\LaTeX ~ ist ein Textverarbeitungsprogramm, das uns in den kommenden
Wochen und Semestern dabei hilft, wissenschaftliche Texte zu verfassen.
\subsection{Basis Befehle}
Damit \LaTeX ~funktioniert, wie es funktionieren soll, müssen bestimmte Befehle
beherrscht werden. Die Befehle dieses Kapitels dienen Hauptsächlich der Formatierung
von Texten und helfen den Code auf ein Minimum von Zeilen zu reduzieren. \\
Jedes Dokument beginnt mit dem Befehl \textbf{\textbackslash documentclass\{...\}}
in die geschweifte Klammer wird eingefügt um welche Dokumentenklasse es sich
handelt. \\
Es gibt folgende Dokumentklassen:
\begin{itemize}
  \item {scrartcl}
  \begin{itemize}
    \item für Artikel
  \end{itemize}
  \item {scrreprt}
  \begin{itemize}
    \item für Berichte
  \end{itemize}
  \item {scrbook}
  \begin{itemize}
    \item für Bücher
  \end{itemize}
\end{itemize}
Die Dokumentenklasse steht stets \textbf{vor der Präambel}. Nachdem die Präambel
abgeschlossen ist, muss in dem Code eine \textit{"Umgebung"} geöffnet werden.
Da wir derzeit nur einfache Textdokumente betrachten, muss mit dem Befehl
\fat{\textbackslash{begin}\{document\}}
eine Umgebung kreiert werden.
\fat{\textsc{Achtung}}: eine geöffnete Umgebung muss immer geschlossen werden.
\subsubsection{Präambel}
Nachdem die Dokumentenklasse festgelegt wurde folgt die \textbf{Präambel}.
hier werden Einstellungen (z.B. Sprache) eingetragen, die das Dokument betreffen.
Die Präambel ist auch der Ort, an dem neue Befehle eingefügt werden können.
Neue Befehle werden mit \fat{\textbackslash newcommand\{...\}\{...\}} festgelegt.
Die erste Klammer beinhaltet den Neuen Befehl, während die zweite
Klammer festlegt, welcher Befehl ersetzt wird.
\newpage
\subsection{Textaufbau}
Wissenschaftliche Texte, wie später eine Bachelor-, bzw. Masterarbeit, benötigen
einen gut gegliederten Aufbau, ein Deckblatt und natürlich ein Inahltsverzeichnis. \\
Folgende Befehle sind hilfreich um längere Texte gut zu gliedern.
\begin{table}
  \centering
  \caption{Gliederung}
  \label{tab:gld}
  \begin{tabular}{l l}
    \toprule
    Input & Output
    \\
    \midrule
    \textbackslash title  & Titel des Dokuments \\
    \textbackslash author  & Autor des Dokuments \\
    \textbackslash maketitle  & generiert einen Titel \\
    \textbackslash tableofcontents  & erstellt ein Inhaltsverzeichnis \\
    \textbackslash section &  beginnt einen neuen Abschnitt \\
    \textbackslash subsection  & beginnt einen Unter-Abschnitt \\
    \textbackslash sub...subsection  & sieht man doch \\
    \bottomrule
  \end{tabular}
\end{table}
\\

\newpage

\section{\LaTeX und die Mathematik}
\subsection{Voreinstellungen}
Damit man mathematische Texte in \LaTeX verfassen kann, müssen zunächst die
richtigen Pakete geladen sein (dies ist hier schon geschehen). \\
Eine Matheumgebung wird mit \fat{\textbackslash begin\{equation\}} geöffnet. Diese
dient Hauptsächlich zum Niederschreiben, da diese direkt nummeriert werden.
alternativ können kürzere Mathe-Umgebungen mit \fat{\$} geöffnet und geschlossen werden.
Dies ist vor allem dann hilfreich, wenn man Formel im Fließtext unterbringen will. \\
Es folgt eine Tabelle mit einer groben Übersicht der Mathematischen Befehle.
\subsection{Befehle}
\begin{table}
  \centering
  \caption{Mathebefehle}
  \label{tab:mat}
  \begin{tabular}{l c c}
    \toprule
    Input & & Output in \$
    \\
    \midrule
    \textbackslash frac\{a\}\{b\} & & $\frac{a}{b}$ \\
    \\
    \textbackslash sqrt\{25\} &  & $\sqrt{25}$ \\
    \\
    \textbackslash int\_\{a\}\^\{b\} & & $\int_{a}^{b}$ \\
    \\
    \textbackslash int\textbackslash limits\_\{a\}\^\{b\}
    & & $\int\limits_{a}^{b}$ \\
    \\
    \textbackslash sum\_\{i=1\}\^\{N\} & & $\sum_{i=1}^{N}$ \\
    \\
    \textbackslash sum\textbackslash limits \_\{a\}\^\{b\}
    & & $ \sum\limits_{i=1}^{N}$ \\
    \\
    \textbackslash prod \_\{i=1\}\^\{N\} & & $\prod_{i=1}^{N}$ \\
    \\
    \textbackslash lim\_\{x\textbackslash to\textbackslash infty\}
    & & $\lim_{x\to\infty}$ \\
    \\
    \{n\textbackslash choose k\} & & ${n\choose k}$ \\
    \\
    \bottomrule
  \end{tabular}
\end{table}
\newpage
Neben den wichtigen Befehlen für Matheoperationen, gibt es auch nützliche Befehle
für Symbole und Sonderzeichen.
\begin{table}
  \centering
  \caption{Symbole\& Sonderzeichen}
  \label{tab:sym}
  \begin{tabular}{l c c l c c} %wo ist der Unterschied zw. l und c????
    \toprule
    Input & & Output in \$ & Input & & Output in \$
    \\
    \midrule
    \textbackslash pm & & $\pm$ & \textbackslash cap & & $\cap$ \\
    \\
    \textbackslash mp & & $\mp$ & \textbackslash cup & & $\cup$ \\
    \\
    \textbackslash cdot & & $\cdot$ &\textbackslash subset & & $\subset$ \\
    \\
    \textbackslash times & & $\times$ &\textbackslash supset & & $\supset$ \\
    \\
    \textbackslash cdots & & $\cdots$ &\textbackslash vee & & $\vee$ \\
    \\
    \textbackslash vdots & & $\vdots$ &\textbackslash wedge & & $\wedge$ \\
    \\
    \textbackslash ddots & & $\ddots$ &\textbackslash in & & $\in$ \\
    \\
    \textbackslash div & & $\div$ &\textbackslash notin & & $\notin$ \\
    \\
    \textbackslash not= & & $\not=$ &\textbackslash perp & & $\perp$ \\
    \\
    \textbackslash equiv & & $\equiv$ &\textbackslash parallel & & $\parallel$\\
    \\
    \textbackslash inftyr & & $\infty$ &\textbackslash forall & & $\forall$ \\
    \\
    \textbackslash partial & & $\partial$ &\textbackslash exists & & $\exists$ \\
    \\
    \textbackslash nabla & & $\nabla$ &\textbackslash Delta & & $\Delta$ \\
    \bottomrule
  \end{tabular}
\end{table}
\\
\\
Während 'normale' Ableitungen wie etwa $y'$ normal in einer Matheumgebung
eingegeben werden können, benötigt man für Ableitungen nach der Zeit
den Befehl \fat{\textbackslash dot x}. Die Anzahl der 'd' gibt an, um die wie vielte
Ableitung es sich handelt. So wird aus \textbackslash dddot z.B.:
$\dddot x$ \\
\section{Listen und Tabellen}
\section{Einbinden von Grafiken}
\section{Fußnoten}

\section{git-Befehle}
\begin{table}
  \centering
  \caption{git}
  \label{tab:git}
    \begin{tabular} {c c}
      \toprule %wofür steht toprule und midrule???????
      Input & Output
      \\
      \midrule
       git status & zeigt den aktuellen Satus an \\
       git log & zeigt Commits an \\
       git add <Datei> & added Datei lokal \\
       git commit -m 'Kommentar' & Kommentar zur Änderung \\
       git pull / push & Up-/ Downdload auf/ von Github \\
       gti rm <Datei> & löscht eine Datei aus Github \\
       git merge master & master branch wird auf den eigenen branch übertragen \\
       git show & zeigt die Änderungen des letzten Commits \\
      \bottomrule
    \end{tabular}
  \end{table}

  \newpage

  \section{Unix-Befehle}
  \begin{table}
    \centering
    \caption{Unix}
    \label{tab:unix}
      \begin{tabular} {c c}
        \toprule
        Input & Output
        \\
        \midrule
        cd & wechselt den Pfad \\
        pwd & zeigt den aktuellen Pfad an \\
        ls & zeigt die aktuellen Dateien an \\
        ls -l & zeigt mehr Informationen über Dateien und Verzeichnisse an \\
        ls -a & zeigt versteckte Dateien an \\
        mkdir <name> & erstellt ein Verzeichnis \\
        touch <name> & erstellt ein Datei \\
        mv <Datei> <Zielpfad> & verschiebt eine Datei \\
        rm & löscht eine Datei \\
        rm -r & löscht ein Verzeichnis \\
        rmdir & löscht ein leeres Verzeichnis \\
        cat <Datei> & gibt Inhalt einer Datei aus \\
        ctrl + C & beendet laufendes Programm \\
        ctrl + D & kann laufende Programme beenden \\
        ctrl + L & leert den Bildschirm (Terminal) \\
        \bottomrule
      \end{tabular}
  \end{table}

\section{Zusatz}
\end{document}
