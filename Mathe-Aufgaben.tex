\documentclass {scrartcl}

\usepackage[aux]{rerunfilecheck}
\usepackage{polyglossia}
\setmainlanguage{german}
\usepackage{amsmath} %unverzichtbare Mathe-Befehle
\usepackage{amssymb} %viele Mathe-Symbole
\usepackage{fontspec}
\usepackage{mathtools} %Erweiterungen für amsmath
\usepackage{unicode-math}
%Noch mehr Pakete
\setmathfont{Latin Modern Math}

\usepackage[unicode]{hyperref}
\usepackage{bookmark}
%Einstellungen hier, z.b. Fonts

\newcommand{\be}{\begin{equation}} %Kurzbefehl für \begin{equation}
\newcommand{\ee}{\end{equation}} %Kurzbefehl für \end{equation}
%Befehle eingefügt um Zeichen zu sparen


\begin{document}
\section{Biot-Savart-Gesetz}
Das Magnetfeld $\vec{B}$ am Ort $\vec{r}$
eines stromdurchlossenen Leiters ergibt sich zu
 \be
    \vec{B}
    (\vec{r})
    =
    \frac{\mu_o}{4\pi}
    \int_V
    \vec{j}
    \vec{(x')}
    \times \frac{\vec{r}-\vec{r'}}{\lvert \vec{r} - \vec{r'}\rvert^3} dV'.
      \ee
  Hierbei bezeichnet $\vec{j}$ die Stromdichte am Ort $\vec{r'}$
  und ${\mu_0}$ die magnetische Feldkonstante.

  \section{Fehlerfortpflanzung}
  \begin{equation}
    \sigma_k =
    \sqrt{\sum_{i=}^N}
    \frac{\delta f}{\delta x_i}
  \end{equation}

\end{document}


%das hast du sehr gut gemacht!
