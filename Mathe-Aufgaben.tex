\documentclass {scrartcl}

\usepackage[aux]{rerunfilecheck}
\usepackage{polyglossia}
\setmainlanguage{german}
\usepackage{amsmath} %unverzichtbare Mathe-Befehle
\usepackage{amssymb} %viele Mathe-Symbole
\usepackage{fontspec}
\usepackage{mathtools} %Erweiterungen für amsmath
\usepackage{unicode-math}
%Noch mehr Pakete
\setmathfont{Latin Modern Math}

\usepackage[unicode]{hyperref}
\usepackage{bookmark}
%Einstellungen hier, z.b. Fonts

\begin{document}
\section{Biot-Savart-Gesetz}
Das Magnetfeld
\vec{B}
am Ort
\vec{r}
eines stromdurchlossenen Leiters
 ergibt sich zu
  \begin{equation}
    \vec{B}
    \vec{(r)}
    =
    \frac{\mu_o}{4\pi}
    \int_V
    \vec{j}
    \vec{(x')}

  \end{equation}
\end{document}
