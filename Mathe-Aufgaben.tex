\documentclass {scrartcl}

\usepackage[aux]{rerunfilecheck}
\usepackage{polyglossia}
\setmainlanguage{german}
\usepackage{amsmath} %unverzichtbare Mathe-Befehle
\usepackage{amssymb} %viele Mathe-Symbole
\usepackage{fontspec}

\usepackage[
  math-style=ISO,
  bold-style=ISO,
  sans-style=italic,
  nabla=upright,
  partial=upright,
]{unicode-math}

\usepackage{mathtools} %Erweiterungen für amsmath
\usepackage[
  math-style=ISO,
  bold-style=ISO,
  sans-style=italic,
  nabla=upright,
  partial=upright,
]{unicode-math} %optionale müssen ergänzt werden
%Noch mehr Pakete
\setmathfont{Latin Modern Math}

\usepackage[unicode]{hyperref}
\usepackage{bookmark}
%Einstellungen hier, z.b. Fonts

\newcommand{\be}{\begin{equation}} %Kurzbefehl für \begin{equation}
\newcommand{\ee}{\end{equation}} %Kurzbefehl für \end{equation}
%Befehle eingefügt um Zeichen zu sparen


\begin{document}
\section{Biot-Savart-Gesetz}
Das Magnetfeld $\vec{B}$ am Ort $\vec{r}$
eines stromdurchlossenen Leiters ergibt sich zu
 \be
    \vec{B}
    (\vec{r})
    =
    \frac{\mu_o}{4\pi}
    \int_V
    \vec{\jmath}
    \vec{(x')}
    \times \frac{\vec{r}-\vec{r'}}{\lvert \vec{r} - \vec{r'}\rvert^3} dV'.
      \ee
  Hierbei bezeichnet $\vec{j}$ die Stromdichte am Ort $\vec{r'}$
  und ${\mu_0}$ die magnetische Feldkonstante.

  \section{Fehlerfortpflanzung}
  \begin{equation}
    \sigma_k =
    \sqrt{\sum_{i=1}^N
    \Biggl(\frac{\partial{f}}{\partial{x_i}} \sigma_i \Biggr)^2}
  \end{equation}
  \section{Maxwell-Gleichungen}

  \begin{align}
    \nabla\cdot\vec{E} &= \frac{\rho}{\epsilon_0}  &  \nabla\cdot\vec{B} &= 0 \\
    \nabla\times\vec{E} &= -\partial_t\vec{B} & \nabla\times\vec{B} &= \mu_0\vec{\jmath}
    +\mu_0\epsilon_0\partial_t\vec{E}
\end{align}
\section{Wellengleichung}
Im Vakuum gelten $\rho=0$ und $\vec{\jmath}=0$, womit sich die Maxwellgleichungen
zu
\begin{align}
  \nabla\cdot\vec{E}&=0 \label{eqn:max1} \\
  \nabla\cdot\vec{B}&=0 \label{eqn:max2} \\
  \nabla\times\vec{E}&=-\partial_t\vec{B} \label{eqn:max3} \\
  \nabla\times\vec{B}&=\mu_0\epsilon_0\partial_t\vec{E} \label{eqn:max4}
\end{align}
reduzieren. Nach erneuter Rotation auf \eqref{eqn:max3} ergibt sich
\be
\nabla\times\bigl(\nabla\times\vec{E}\bigr)=\nabla\times\bigl(-\partial_t\vec{B}\bigr)
\label{eqn:rot}
\ee
Nach dem Satz von Schwarz lassen sich die partiellen Ableitungen vertauschen, was zu
\be
\nabla\times\bigl(\nabla\times\vec{E}\bigr)=-\partial_t\bigl(\nabla\times\vec{B}\bigr)
\label{eqn:schwarz}
\ee
führt. Wir setzen auf der rechten Seite \eqref{eqn:max4} ein:
\be
\nabla\times\bigl(\nabla\vec{E}\bigr)=-\mu_0\epsilon_0\symup{\partial_t^2}\vec{E}\label{eqn:ein}
\ee
aus der linken Seite wird mit
\be
\nabla\times\bigl(\nabla\vec{E}\bigr)=\nabla\cdot\bigl(\nabla\vec{E}\bigr)-\Delta
\vec{E} \label{eqn:2}
\ee
\newpage
und ausnutzen von \eqref{eqn:max1}
\be
-\Delta\vec{E}=-\mu_0\epsilon_0\symup{\partial_t^2}\vec{E}.
\label{eqn:13}
\ee
Dies ist die Wellengleichung für das elektrische Feld,
in der sich die Lichgeschwindigkeit
\be
c=\frac{1}{\sqrt{\mu_0\epsilon_0}}\label{eqn:14}
\ee
identifizieren lässt. Damit können wir
\be
\Bigl(\Delta-\frac{1}{c^2}\frac{\symup{\partial^2}}{\symup{\partial}t^2}\Bigr)
\vec{E}=0 \label{eqn:15}
\ee
schreiben.
\section{Wellengleichung}
Ebene Welle:
\be
\nabla^2 A -\frac{1}{c^2}\frac{\symup{\partial^2}}{\symup{\partial}t^2}A=0
\label{eqn:16}
\ee
eine Lösung:
\be
A=A_0\exp(\bigl(\symup{i}\bigl(\symbf{k} \symbf{x}-\omega t\bigr)\bigr)) \label{eqn:17}
\ee
Gruppen- und Phasengeschwindigkeit:
\begin{align}
  v_{Gr}&=\frac{\symup{\partial}\omega}{\symup{\partial}k} &
  v_{Ph}&=\frac{\omega}{k}
\end{align}
\end{document}
%das hast du sehr gut gemacht!
