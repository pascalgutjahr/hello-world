%Mathematischer Header
\usepackage{scrhack}
\usepackage[aux]{rerunfilecheck}

\usepackage{polyglossia}
\setmainlanguage{german}

\usepackage{amsmath}
\usepackage{amssymb}
\usepackage{mathtools}

\usepackage{fontspec}

\usepackage[
  math-style=ISO,
  bold-style=ISO,
  sans-style=italic,
  nabla=upright,
  partial=upright,
]{unicode-math}
\usepackage[
locale=DE,
separate-uncertainty=true, % Immer Fehler mit ±
per-mode=symbol-or-fraction, % m/s im Text, sonst \frac
% alternativ:
% per-mode=reciprocal, % m s^{-1}
% output-decimal-marker=., % . statt , für Dezimalzahlen
]{siunitx}

\usepackage[section, below]{placeins}
\usepackage[
  labelfont=bf,        % Tabelle x: Abbildung y: ist jetzt fett
  font=small,          % Schrift etwas kleiner als Dokument
  width=0.9\textwidth, % maximale Breite einer Caption schmaler
  format=plain
  indention=1em % Abbildung sticht links etwas hervor
]{caption}
\usepackage{graphicx}
\usepackage{grffile}
\usepackage{subcaption}
\usepackage{booktabs}
\usepackage{float}
\floatplacement{figure}{htbp}
\floatplacement{table}{htpb}
%Einstellungen hier, z.b. Fonts

\usepackage[unicode]{hyperref}
\usepackage{bookmark}

\newcommand{\be}{\begin{equation}} %Kurzbefehl für \begin{equation}
\newcommand{\ee}{\end{equation}}
