\documentclass{scrartcl}
%Mathematischer Header
\usepackage{scrhack}
\usepackage[aux]{rerunfilecheck}

\usepackage{polyglossia}
\setmainlanguage{german}

\usepackage{amsmath}
\usepackage{amssymb}
\usepackage{mathtools}

\usepackage{fontspec}

\usepackage[
  math-style=ISO,
  bold-style=ISO,
  sans-style=italic,
  nabla=upright,
  partial=upright,
]{unicode-math}
\usepackage[
locale=DE,
separate-uncertainty=true, % Immer Fehler mit ±
per-mode=symbol-or-fraction, % m/s im Text, sonst \frac
% alternativ:
% per-mode=reciprocal, % m s^{-1}
% output-decimal-marker=., % . statt , für Dezimalzahlen
]{siunitx}

\usepackage[section, below]{placeins}
\usepackage{caption}
\usepackage{graphicx}
\usepackage{grffile}
\usepackage{subcaption}
\usepackage{booktabs}
\usepackage{float}
\usepackage{figure}{htbp}
\usepackage{table}{htpb}
%Einstellungen hier, z.b. Fonts

\usepackage[unicode]{hyperref}
\usepackage{bookmark}

\newcommand{\be}{\begin{equation}} %Kurzbefehl für \begin{equation}
\newcommand{\ee}{\end{equation}}


\begin{document}
\section{Aufgabe 1}
In \ref{fig:plt1} sieht man einen schönen Plot.
\begin{figure}
  \centering
  \includegraphics[width=\textwidth]{plots/plot1.pdf}
  \caption{ein schöner Plot.}
  \label{fig:plt1}
\end{figure}
\\
Hier kann man noch einiges über den gezeigten Plot erzählen,
wenn man genug zu erzählen hat.
\newpage
\section{Aufgabe 2}
In \ref{fig:2} sieht man 2 schöne plots. Abbildung \ref{fig:2a}
ist der schönste aller plots.
\begin{figure}
  \centering
  \begin{subfigure}{0.48\textwidth}
    \centering
    \includegraphics[width=\textwidth]{plots/plot2.pdf}
    \caption{Der schönste Plot.}
    \label{fig:2a}
  \end{subfigure}
  \begin{subfigure}{0.48\textwidth}
    \centering
    \includegraphics[width=\textwidth]{plots/plot3.pdf}
    \caption{schöner Plot.}
    \label{fig:2b}
  \end{subfigure}
  \caption{Zwei Plots, Abbildung \subref{fig:2a}: schönster Plot.}
  \label{fig:2}
\end{figure}
\\
Auch hierzu könnte man mehr erzählen, wenn man lustig ist.

\end{document}
