\documentclass[bibliography=totoc]{scrartcl}

\usepackage[aux]{rerunfilecheck}

\usepackage{polyglossia}


\setmainlanguage{german}

\usepackage{fontspec}

\usepackage[style=alphabetic]{biblatex}
\addbibresource{lit.bib}  %nach polyglossia

\usepackage[unicode]{hyperref}
\usepackage{bookmark}

\begin{document}

\section{Theorie}

Physik macht Spaß \cite[1--3]{deGennes}

\section{Aufbau und Durchführung}

informatik ist scheiße \cite[10]{magnet}

\section{Diskussion}

das wird hier ganz klar bestätigt \cite[20--21]{kent}

dies sagt auch Martin\cite[5]{Martin}

\printbibliography

\end{document}

%um das Literaturverzeichnis nun korrekt anzuzeigen,
%sind folgende Befehle notwendig:
%lualatex biblatex.tex
%biber biblatex.bcf
%lualatex biblatex.tex
%das dies 3 Befehle sind, bietet sich hier ein Makefile an!
